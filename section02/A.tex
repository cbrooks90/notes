f(x,y) = ax^2 + bxy + cy^2

Primitive means gcd(a,b,c)=1
m is represented by f(x,y) means
  m=f(x,y) for integers x,y
m is properly represented if gcd(x,y)=1.

f(x,y) and g(x,y) are equivalent if there exist
p,q,r,s such that
 - f(x,y) = g(px+qy,rx+sy)
 - ps-qr = +-1

This means the matrix (p q, r s) is in GL(2,Z).
Equivalence of forms is an equivalence relation:

Reflexivity via the identity matrix, symmetric via inverse matrix, transitivity via matrix product. These work because det(id) = 1, det(A^-1) = det(A) if det(A) = +-1, and det(AB)=det(A)det(B).

This may be useful: If f(x,y) = g(px+qy,rx+sy), then f((sx-qy)/(ps-qr), (-rx+py)/(ps-qr)) = g(x,y).

In fact, we can explicitly write the coefficients of g in this case. If f(x,y)=ax^2+bxy+cy^2, then
  g(px+qy, rx+sy) = f(x,y), or equivalently
  g(x,y) = f((sx-qy)/D, (-rx+py)/D) (D is the discriminant)
  g(x,y) = (as^2-brs+cr^2)x^2
           + (b(ps+qr)-2aqs-2cpr)xy
           + (aq^2-bpq+cp^2)y^2 (since D^2=1).

Equivalent forms represent the same numbers. Suppose f ~ g and exist a,b such that m=f(a,b). Then m=g(pa+qb, ra+sb) and vice versa (using the inverse). A proper representation is one such that m=f(a,b) and gcd(a,b)=1; A representation is improper if gcd(a,b)>1.

Is it possible to get an improper representation from a proper one? No. Suppose f ~ g via integers p,q,r,s with pq-rs=+-1, and that m = g(pa+qb, ra+sb) = f(a,b) with gcd(a,b)=1. We will show that gcd(pa+qb, ra+sb)=1.

We have s(pa+qb)-q(ra+sb)=+-a and
       -r(pa+qb)+p(ra+sb)=+-b.
Since gcd(a,b)=1, there are integers n,m such that n(+-a)+m(+-b)=1, where the signs match the equations we just wrote. Then after substituting,
        n(s(pa+qb)-q(ra+sb)) + m(-r(pa+qb)+p(ra+sb)) = 1,
which can be rearranged to
        (ns-mr)(pa+qb) + (mp-nq)(ra+sb) = 1.
This is only possible if gcd(pa+qb, ra+sb)=1.

Since this argument is the same with the inverse transformation, we have that the set of integers properly represented by f is an equivalence invariant.

Another such invariant is the discriminant.

Suppose f(x,y)=ax^2+bxy+cy^2 has discriminant D = b^2-4ac. Then if g ~ f via p,q,r,s, as shown above we can write
  g(x,y) = (as^2-brs+cr^2)x^2
           + (b(ps+qr)-2aqs-2cpr)xy
           + (aq^2-bpq+cp^2)y^2
which has discriminant
D'=(b(ps+qr)-2(aqs+cpr))^2 - 4(as^2-brs+cr^2)(aq^2-bpq+cp^2).
  = (bqr)^2 + (bps)^2 + 2b^2pqrs
     + 4(aqs)^2 + 4(cpr)^2 + 8acpqrs
     - 4abrsq^2 - 4bcrsp^2 - 4abpqs^2 -4bcpqr^2

     -4(aqs)^2 - 4b^2pqrs - 4(cpr)^2
      + 4abpqs^2 + 4abq^2rs
      - 4acp^2s^2 - 4acq^2r^2
      + 4bcp^2rs + 4bcpqr^2

Combining,
D' = (bps)^2 + (bqr)^2 + 2b^2pqrs - 4b^2pqrs
      + 4(aqs)^2 - 4(aqs)^2
      + 4(cpr)^2 - 4(cpr)^2
      - 4abpqs^2 + 4abpqs^2
      - 4bcpqr^2 + 4bcpqr^2
      - 4abq^2rs + 4abq^2rs
      - 4bcp^2rs + 4bcp^2rs
      + 8acpqrs
      - 4acp^2s^2
      - 4acq^2r^2
   = (bps)^2 + (bqr)^2 - 2b^2pqrs
      + 8acpqrs - 4acp^2s^2 - 4acq^2r^2
   = b^2((ps)^2 + (qr)^2 - 2pqrs)
      - 4ac((ps)^2 + (qr)^2 - 2pqrs)
   = (b^2 - 4ac)((ps)^2 + (qr)^2 - 2pqrs)
   = (b^2 - 4ac)(ps - qr)^2
   = D(ps - qr)^2 = D.

With these notions we can begin to see the relation to x^2+ny^2 problem (the bridge being quadratic residues):

If D=0,1 mod 4 and m is an odd integer with gcd(D, m)=1, then there is a primitive quadratic form of discriminant D properly representing m <=> D is a quadratic residue mod m.

I think Lemma 2.3 should be inlined, since it only appears in one other place and is fairly trivial.

Suppose f(p,q)=m=ap^2+bpq+cq^2 with f primitive and gcd(p,q)=1. There exists s,t such that sp-tq=1. Then f(px+ty, qx+sy)=a(px+ty)^2+b(px+ty)(qx+sy)+c(qx+sy)^2
 = x^2(ap^2+bpq+cq^2) + xy(2apt+bps+bqt+2cqs) + y^2(at^2+bst+cs^2)
 = mx^2 + b'xy + f(t,s)y^2
Then D=(b')^2 - 4mf(t,s)^2, so D is a quadratic residue mod m. I'm not sure about gcd(D,m)=1 here. If gcd(D,m) is not 1 is it impossible? 25 = 0^2 mod 5 and

Conversely, if D is a QR mod m, then for some i, D=i^2 mod m. This is where we require D to be 0 or 1 mod 4, since otherwise D cannot be a discriminant. If D=i mod 2 then D-i^2 = 0 mod 4. If D \neq i mod 2 then D-(i+m)^2 = 0 mod 4. Let j be i or i+m so that D=j^2 mod 4 in either case. Since gcd(m,4)=1, we can find c so that D=j^2-4cm. Then mx^2+jxy+cy^2 does the job. Think about primitiveness of both directions.

As we've done previously, we consider discriminants which are multiples of 4 since we don't lose anything and we can relate this lemma to our previous work. This is made explicit in the corollary:

If $m$ is a positive, odd integer, there is a primitive quadratic form of discriminant $-4n$ representing $m \iff (-4n/m)=1$. Note that we no longer require the representation to be proper. Is it impossible to keep proper representation?

Reduced positive definite quadratic forms

A primitive positive definite form $ax^2+bxy+cy^2$ is \emph{reduced} if $|b|\leq a\leq c$; when either is an equality, we also require $b\geq 0$. We want positive definiteness because for any positive definite quadratic form, there is a unique reduced form properly equivalent to it.

The criterion for positive-definiteness appears naturally if you try and complete the square.

\begin{align*}
  f(x,y) &= ax^2 + bxy + cy^2\\
 af(x,y) &= a^2x^2 + abxy + acy^2\\
         &= a^2x^2 + abxy + b^2y^2/4 - b^2y^2/4 + acy^2\\
         &= (ax + (b/2)y)^2 + acy^2-b^2y^2/4\\
         &= (ax + (b/2)y)^2 + (ac-b^2/4)y^2\\
4af(x,y) &= (2ax+by)^2 + (4ac-b^2)y^2\\
         &= (2ax+by)^2 - Dy^2.\\
\end{align*}

$D<0\implies a\neq 0$. If $a>0$, $f(x,y)\geq 0\forall x,y\in\mathbf{Z}$. Can $f(x,y)$ ever be zero?

$0 = (2ax+by)^2 + (-D)y^2$
which means $y=0$ and $(2ax)^2 = 0$, so $x=0$. If we take the definitions of ``represent'' and ``properly represent`` literally, we can exclude zero by saying
\[D<0, a>0\implies f(x,y) > 0\forall x,y\in\mathbf{Z}, \gcd(x,y)=1\]
or equivalently: If $D<0$ and $a>0$, $m>0$ for all $m$ properly represented by $f$. (This requires us to remember that $\gcd(0,0)=0$ since ``greatest'' common divisor refers to maximality in the divisibility lattice.)

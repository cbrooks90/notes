f(x,y) = ax^2 + bxy + cy^2

Primitive means gcd(a,b,c)=1
m is represented by f(x,y) means
  m=f(x,y) for integers x,y
m is properly represented if gcd(x,y)=1.

f(x,y) and g(x,y) are equivalent if there exist
p,q,r,s such that
 - f(x,y) = g(px+qy,rx+sy)
 - ps-qr = +-1

This means the matrix (p q, r s) is in GL(2,Z).
Equivalence of forms is an equivalence relation:

Reflexivity via the identity matrix, symmetric via inverse matrix, transitivity via matrix product. These work because det(id) = 1, det(A^-1) = det(A) if det(A) = +-1, and det(AB)=det(A)det(B).

This may be useful: If f(x,y) = g(px+qy,rx+sy), then f((sx-qy)/(ps-qr), (-rx+py)/(ps-qr)) = g(x,y).

In fact, we can explicitly write the coefficients of g in this case. If f(x,y)=ax^2+bxy+cy^2, then
  g(px+qy, rx+sy) = f(x,y), or equivalently
  g(x,y) = f((sx-qy)/D, (-rx+py)/D) (D is the discriminant)
  g(x,y) = (as^2-brs+cr^2)x^2
           + (b(ps+qr)-2aqs-2cpr)xy
           + (aq^2-bpq+cp^2)y^2 (since D^2=1).

Equivalent forms represent the same numbers. Suppose f ~ g and exist a,b such that m=f(a,b). Then m=g(pa+qb, ra+sb) and vice versa (using the inverse). A proper representation is one such that m=f(a,b) and gcd(a,b)=1; A representation is improper if gcd(a,b)>1.

Is it possible to get an improper representation from a proper one? No. Suppose f ~ g via integers p,q,r,s with pq-rs=+-1, and that m = g(pa+qb, ra+sb) = f(a,b) with gcd(a,b)=1. We will show that gcd(pa+qb, ra+sb)=1.

We have s(pa+qb)-q(ra+sb)=+-a and
       -r(pa+qb)+p(ra+sb)=+-b.
Since gcd(a,b)=1, there are integers n,m such that n(+-a)+m(+-b)=1, where the signs match the equations we just wrote. Then after substituting,
        n(s(pa+qb)-q(ra+sb)) + m(-r(pa+qb)+p(ra+sb)) = 1,
which can be rearranged to
        (ns-mr)(pa+qb) + (mp-nq)(ra+sb) = 1.
This is only possible if gcd(pa+qb, ra+sb)=1.

Since this argument is the same with the inverse transformation, we have that the set of integers properly represented by f is an equivalence invariant.

Another such invariant is the discriminant.

Suppose f(x,y)=ax^2+bxy+cy^2 has discriminant D = b^2-4ac. Then if g ~ f via p,q,r,s, as shown above we can write
  g(x,y) = (as^2-brs+cr^2)x^2
           + (b(ps+qr)-2aqs-2cpr)xy
           + (aq^2-bpq+cp^2)y^2
which has discriminant
D'=(b(ps+qr)-2(aqs+cpr))^2 - 4(as^2-brs+cr^2)(aq^2-bpq+cp^2).
  = (bqr)^2 + (bps)^2 + 2b^2pqrs
     + 4(aqs)^2 + 4(cpr)^2 + 8acpqrs
     - 4abrsq^2 - 4bcrsp^2 - 4abpqs^2 -4bcpqr^2

     -4(aqs)^2 - 4b^2pqrs - 4(cpr)^2
      + 4abpqs^2 + 4abq^2rs
      - 4acp^2s^2 - 4acq^2r^2
      + 4bcp^2rs + 4bcpqr^2

Combining,
D' = (bps)^2 + (bqr)^2 + 2b^2pqrs - 4b^2pqrs
      + 4(aqs)^2 - 4(aqs)^2
      + 4(cpr)^2 - 4(cpr)^2
      - 4abpqs^2 + 4abpqs^2
      - 4bcpqr^2 + 4bcpqr^2
      - 4abq^2rs + 4abq^2rs
      - 4bcp^2rs + 4bcp^2rs
      + 8acpqrs
      - 4acp^2s^2
      - 4acq^2r^2
   = (bps)^2 + (bqr)^2 - 2b^2pqrs
      + 8acpqrs - 4acp^2s^2 - 4acq^2r^2
   = b^2((ps)^2 + (qr)^2 - 2pqrs)
      - 4ac((ps)^2 + (qr)^2 - 2pqrs)
   = (b^2 - 4ac)((ps)^2 + (qr)^2 - 2pqrs)
   = (b^2 - 4ac)(ps - qr)^2
   = D(ps - qr)^2 = D.

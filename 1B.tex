Here are some examples of the ``Reciprocity'' and ``Descent'' steps for the x^2 + y^2 problem. Specifically, I would like clarify why all parts of the two steps are necessary.

Take p = 5.

Then x^4 - 1 = 0 mod 5 for all x != 0.

x^4 - 1 = (x^2 + 1)(x^2 - 1) = 0.

Can we find x such that x^2 - 1 != 0 mod 5? Sure, x = 2 or 3.

x = 2 => 4 + 1 is divisible by 5.

x = 3 => 9 + 1 is divisible by 5.

In the second case we need to use descent.

Once we reduce 9 mod 5 we're done. Is this always the case?

No: 5 divides 3^2 + 4^2.

Do we ever get a case where a < p/2 and b = 1 but p != a^2 + b^2?

a^2 + 1 < p^2/4 + 1. Can p divide this without equaling?

p^2/4 + 1 > 2p
p^2 + 4 > 8p
Take p = 13

x^12 - 1 = (x^6 + 1)(x^6 - 1) = 0 for x != 0.

x^6 != 1 mod 13?

x = 2, so 13 divides 8^2 + 1. Once we reduce, we get

13 divides (-5)^2 + 1 = 26 but 13 != 26. So we need to descend.

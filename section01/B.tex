\documentclass[b5paper,12pt,oneside,openright]{memoir}
\usepackage{amsmath, amsthm, amssymb}
\usepackage{enumitem}
\usepackage{color}

\setlength{\uppermargin}{22mm}
\setlength{\spinemargin}{22mm}
\setlength{\textheight}{206mm}
\setlength{\textwidth}{132mm}
\setlength{\oddsidemargin}{0mm}
\fixthelayout

\begin{document}
\section{Descent is necessary}
Here are some concrete calculations which follow the proof of the reciprocity step. Take $p = 5$. Then $x^4 - 1 = 0 \mod 5$ for all $x \neq 0$. $x^4 - 1 = (x^2 + 1)(x^2 - 1) = 0$. At this point we need to find $n$ such that $n^2 - 1 \neq 0 \mod 5$, so take $n = 2$ or 3.

\begin{align*}
n = 2 &\implies 5\mid 2^2 + 1\\
n = 3 &\implies 5\mid 3^2 + 1\\
\end{align*}

In the second case we might need to use descent. However, $3 > p/2$ so we replace 3 with $-2$ and we're done. Is this always the case? In other words, do we ever get a case where $a < p/2$ and $b = 1$ but $p \neq a^2 + b^2$?

Most likely---I think I just chose $p$ too small. After reducing, \[a^2 + 1 < p^2/4 + 1.\] Can $p$ divide this properly? Let's try:

\begin{align*}
p^2/4 + 1 &> 2p\\
p^2 + 4 &> 8p\\
\end{align*}

$p = 13$ satisfies the last inequality. Starting over, $x^{12} - 1 = (x^6 + 1)(x^6 - 1) = 0$ for $x \neq 0$. Let's find $n$ such that $n^6 \neq 1 \mod 13$. $n = 2$ works, so $13\mid 8^2 + 1$.

After reducing, we get $13 \mid (-5)^2 + 1$ but $13 \neq 26$ and descent is necessary in this case. At some point I will write an implementation of the descent step using the formulas in the book, but for now let's consider the proof.

\section{Proof of the descent step}

Proof by descent is clear to me in easy examples, for instance when proving $\sqrt{2}$ is irrational. In the proof of the descent step for $x^2+y^2 = p$, the contradiction was a little bit trickier for me, hence the ensuing verbosity.

Suppose $a^2 + b^2 = m\cdot p$ for an odd prime $p$, $\gcd(a, b)=1$.

The first step is to remove from both sides all prime factors of $m$ which can be written as a sum of squares. Lemma 1.4 tells us that we will still be left with a sum of squares on the left side. Let's write this as $a_1^2+b_1^2=m_1\cdot p$; we know $m_1<m$ has no prime factors expressible as a sum of squares.

Now we make the assumption which will lead to a contradiction: Assume there is a prime $q_1$ dividing $m_1$. $q_1$ cannot be written as a sum of squares and $q_1<p$ because $a^2+b^2<p^2/4$ and $a_1^2+b_1^2<a^2+b^2$.

At this point we start the process again: $q_1\mid a_1^2+b_1^2$. Reduce $a_1$ and $b_1$ modulo $q_1$, so that $a_2<q_1/2$, $b_2<q_1/2$ and $a_2^2+b_2^2 = m_2\cdot q_1$. $m_2$ must have a prime factor since $q_1$ cannot be written as a sum of squares. Choose a prime $q_2$ dividing $m_2$; we have $q_2<q_1$ and we repeat.

Continuing in this way gives us an infinte descending sequence of positive primes, so our assumption that we could choose a prime factor $q_1$ of $m_1$ was flawed. $m_1$ must be 1, so in fact $p=a_1^2+b_1^2$.

\end{document}

\documentclass[b5paper,12pt,oneside,openright]{memoir}
\usepackage{amsmath, amsthm, amssymb}
\usepackage{enumitem}
\usepackage{color}

\nonzeroparskip

\setlength{\uppermargin}{22mm}
\setlength{\spinemargin}{22mm}
\setlength{\textheight}{206mm}
\setlength{\textwidth}{132mm}
\setlength{\oddsidemargin}{0mm}
\fixthelayout

\newtheorem{thm}{Theorem}
\newtheorem*{thm*}{Theorem}

\DeclareMathOperator{\sgn}{sgn}
\newcommand{\jacobi}[2]{\left(\frac{#1}{#2}\right)}
\newcommand{\Zmod}[1]{\mathbf{Z}/#1\mathbf{Z}}

\begin{document}
\section{The relation between $p\mid x^2+ny^2$ and quadratic reciprocity}
Suppose $x^2 + ny^2 = kp$ for some $k$. Assume there is a common divisor of $y$ and $p$ which is prime; since $p$ is prime, it must be $p$.
Then $p\mid x^2$ which implies $p\mid x$, i.e. $p$ is a common divisor of $x$ and $y$.
This tells us $\gcd(x,y)=1 \implies \gcd(y,p)=1$.

In particular, $\gcd(x,y)=1\implies y$ is invertible modulo $p$ and
\[(xy^{-1})^2\equiv -n\bmod p,\]
so $-n$ is a quadratic residue modulo $p$.

On the other hand, if $-n\equiv x^2\bmod p$ for some $x$, $p \mid x^2+n\cdot 1^2$ (and $\gcd(x,1)=1$). This proves
\[\jacobi{-n}{p}=1 \iff p\mid x^2+ny^2,\ \gcd(x,y)=1.\]

Consider the conditions on $p$ in 1.1: The moduli for $n=1$ and 2 are already multiples of 4, and $p\equiv1\bmod 3$ is equivalent to $p\equiv 1, 4, 7, 10\bmod 12$. We know 4 and 10 are not coprime with 12 so we get $p\equiv 1, 7\bmod 12$ (and this agrees with Euler's conjecture about $(-3/p)$). It seems like the 4 essentially comes from $(-1/p)$ and is the smallest factor needed to write the conditions on $p$ in a unified way for all $n$.

\section{Computing $\jacobi{-1}{p}$ and $\jacobi{2}{p}$}

I took a small break to prove the formulas for $(-1/p)$ and $(2/p)$. These follow from Euler's criterion.

$a=-1$ yields
\[\jacobi{-1}{p} = (-1)^\frac{p-1}{2} =
  \begin{cases}
     1 &\quad p\equiv 1\bmod 4\\
    -1 &\quad p\equiv 3\bmod 4.
  \end{cases}
\]

For $a=2$, we can use the simple proof in Serre if we know that an algebraic closure (or splitting field) exists. For $a=3$, Serre's proof works almost exactly the same since if $\beta$ is a primitive 12th root of unity, $\beta^4-\beta^2+1=0$ so $\beta^2 + \beta^{-2} = 1$. Thus if $x=\beta + \beta^{-1}$, $x^2 = 3$ and the proof procedes the same way.

Here is an elementary proof for $a=2$: Fix a positive odd number $n = 2k+1$ and denote $\{i\in\mathbf{N} : 1\leq i\leq k\}$ by $[k]$.
Define $f:[k]\to[k]$ by
\[f(i) =
  \begin{cases}
    \frac{i}{2} &i\text{ even}\\
    \frac{n-i}{2} &i\text{ odd}.
  \end{cases}
\]
$f$ is surjective since
\begin{align*}
  i\leq k/2 &\implies 2i\leq k\text{ and }f(2i) = i\\
  i > k/2 &\implies n-2i < n-k = k+1\text{ and }f(n-2i) = i.
\end{align*}
Since the domain and codomain are equal, $f$ is a bijection. Thus,
\[\prod_{i=1}^k f(i) = k!\quad\implies\quad\prod_{i=1}^k (-1)^i\cdot 2\cdot f(i) = (-1)^\frac{k^2 + k}{2}\cdot 2^k\cdot k!.\]
Finally, $(-1)^i\cdot 2\cdot f(i)\equiv i\bmod n$, so
\[(-1)^\frac{k^2 + k}{2}\cdot 2^k\cdot k!\equiv k!\bmod n.\]
When $n$ is prime, we can cancel $k!$ and substitute $k=\frac{n-1}{2}$ to get
\[2^\frac{n-1}{2}\equiv (-1)^\frac{n^2 - 1}{8}\bmod n\]
which proves
\[\left(\frac{2}{n}\right) =
  \begin{cases}
     1 &\quad n\equiv\pm 1\bmod 8\\
    -1 &\quad n\equiv\pm 3\bmod 8.
  \end{cases}
\]

It looks like there are some group theoretic methods which are similar to Serre's method but without using splitting fields. Look into this more.

\section{$\jacobi{p^*}{q}=\jacobi{q}{p}\iff \jacobi{p}{q}\jacobi{q}{p}=(-1)^{(p-1)(q-1)/4}$}

For an odd prime $p$, define $p^*$ to be $(-1)^\frac{p-1}{2} p$. In other words,
\[p^*=
\begin{cases}
  p&p\equiv 1\bmod 4\\
  -p&p\equiv 3\bmod 4.\\
\end{cases}
\]

Assume quadratic reciprocity, and consider $(p^*/q)$ for distinct primes $p, q$. If $p\equiv 1\bmod 4$,
\[\jacobi{p^*}{q} = \jacobi{p}{q} = (-1)^{(q-1)\cdot\frac{(p-1)}{4}} = \jacobi{q}{p}.\]
If $p\equiv 3\bmod 4$,
\begin{align*}
  \jacobi{p^*}{q} &= \jacobi{-p}{q}\\
                  &= \jacobi{-1}{q}\jacobi{p}{q}\\
                  &= \jacobi{-1}{q}(-1)^{\frac{q-1}{2}\cdot\frac{p-1}{2}}\jacobi{q}{p}\\
                  &= \jacobi{-1}{q}\jacobi{-1}{q}^\frac{p-1}{2}\jacobi{q}{p}\\
                  &= \jacobi{-1}{q}^{1+\frac{p-1}{2}}\jacobi{q}{p} = \jacobi{q}{p}
\end{align*}
since $1+\frac{p-1}{2}$ is even when $p\equiv 3\bmod 4$. In both cases $(p^*/q)=(q/p)$.

Conversely, assume $(p^*/q) = (q/p)$. Then
\begin{align*}
  \jacobi{p}{q}\jacobi{q}{p} &= \jacobi{p}{q}\jacobi{p^*}{q}\\
                             &= \jacobi{p}{q}\jacobi{p}{q}\jacobi{-1}{q}^\frac{p-1}{2}\\
                             &= 1\cdot\jacobi{-1}{q}^\frac{p-1}{2}\\
                             &= \left((-1)^\frac{q-1}{2}\right)^\frac{p-1}{2}\\
                             &= (-1)^{(p-1)(q-1)/4}
\end{align*}
which is the usual statement of quadratic reciprocity.

This shows the equivalence of $(p^*/q) = (q/p)$ and quadratic reciprocity.

\section{$\jacobi{p^*}{q} = 1 \iff\exists b$ odd s.t. $p\equiv\pm b^2\bmod 4q$}

If $(p^*/q) = 1$, there exists $b$ such that $p^*\equiv b^2\bmod q$. We can assume $b$ is odd by adding $q$ if necessary. Consequently, $b^2\equiv 1\bmod 4$.

If $p\equiv 1\bmod 4$, $p\equiv b^2\bmod q\implies q\mid p-b^2$. Together with $4\mid p-b^2$, we have $p\equiv b^2\bmod 4q$.


If $p\equiv 3\bmod 4$, $-p\equiv b^2\bmod q\implies q\mid p+b^2$. Together with $4\mid p+b^2$, we have $p\equiv -b^2\bmod 4q$.

Conversely, suppose there is $b$ odd such that $p=b^2\bmod 4q$. Then
\begin{align*}
p=b^2\bmod 4&\implies p=1\bmod 4\implies p^*=p,\text{ and}\\
p=b^2\bmod q&\implies\jacobi{p}{q}=1\implies\jacobi{p^*}{q}=1.
\intertext{If there is $b$ odd such that $p=-b^2\bmod 4q$, then}
p=-b^2\bmod 4&\implies p=3\bmod 4\implies p^*=-p,\text{ and}\\
p=-b^2\bmod q&\implies\jacobi{-p}{q}=1\implies\jacobi{p^*}{q}=1.
\end{align*}

\section{The Jacobi symbol as a character}

Quadratic reciprocity for the Jacobi symbol can be written
\[\jacobi{n}{m} = (-1)^{\frac{n-1}{2}\cdot\frac{m-1}{2}} \jacobi{m}{n}.\]
There is an asymmetry with reciprocity in the case when $m$ is not prime.

In the following, we fix a nonzero integer $D\equiv 0, 1\bmod 4$. Define a map
\[\chi:(\mathbf{Z}/D\mathbf{Z})^* \mapsto \{\pm 1\}\]
by the rule
\[[i]\mapsto(D/i)\text{ if $i$ is a positive odd integer.}\]

First we show that the map is well-defined. We need the following lemma.

Lemma: If $k\equiv 0,1\bmod 4$ and $i,j$ are positive odd integers such that $i\equiv j\bmod k$, then $(k/i) = (k/j)$.

Proof:
\begin{enumerate}[label={Case \arabic*:}]
\item $k\equiv 1\bmod 4$, $k>0$. $\frac{k-1}{2}$ is even, so quadratic reciprocity and $(i/k) = (j/k)$ yield
\[\jacobi{k}{i}=\jacobi{i}{k}=\jacobi{j}{k}=\jacobi{k}{j}.\]

\item $k\equiv 1\bmod 4$, $k<0$. Using $(k/i) = (-1/i)(-k/i)$ and quadratic reciprocity we get
\[\jacobi{k}{i}=(-1)^{\frac{i-1}{2}\cdot\left(1-\frac{k+1}{2}\right)}\jacobi{i}{-k}=\jacobi{j}{-k}\]
since $1-\frac{k+1}{2}$ is even. Using the same steps, we show
\[\jacobi{j}{-k} = \jacobi{k}{j}.\]

\item $k\equiv 0\bmod 4$, $k\neq 0$, write $k = 2^n \sgn(k)\cdot k^\prime$  where $2 \nmid k^\prime$ and $k^\prime>0$.
\begin{align*}
\jacobi{k}{i}&=\jacobi{2^n \sgn(k) k^\prime}{i}\\
&=\jacobi{2}{i}^n \jacobi{\sgn(k)}{i}\jacobi{k^\prime }{i}.\\
\intertext{Quadratic reciprocity and $(i/k^\prime) = (j/k^\prime)$ give}
&=\jacobi{2}{i}^n \jacobi{\sgn(k)}{i} (-1)^{\frac{k^\prime-1)}{2}\left(\frac{i+j}{2} - 1\right)}\jacobi{k^\prime }{j}.
\intertext{But since $i = j + 4c$ for some integer $c$ (remember $k\equiv 0\bmod 4$),}
(i+j)/2 - 1 &= (2j + 4c)/2 - 1 = j - 1 + 2c\\
\intertext{which is even. Thus,}
\jacobi{k}{i}&=\jacobi{2}{i}^n \jacobi{\sgn(k)}{i}\jacobi{k^\prime }{j}\\
&=\jacobi{2}{i}^n\jacobi{\sgn(k)}{i}\jacobi{\sgn(k)}{j}\jacobi{\sgn(k)k^\prime}{j}\\
&=\jacobi{2}{i}^n\jacobi{\sgn(k)k^\prime}{j}
\end{align*}
since $i\equiv j\bmod 4$.

Finally, to put the powers of two back, we write $i = j + m\cdot k^\prime\cdot 2^n$ and consider the following cases.

$n>2$: We have $i-j$ divisible by 8, so $(2/i) = (2/j)$.

$n=2$: $(2/i)^2 = (2/j)^2$ is immediate.

In both cases,
\[\jacobi{2}{i}^n = \jacobi{2}{j}^n,\]
so we can continue
\begin{align*}
\jacobi{k}{i}&=\jacobi{2}{j}^n\jacobi{\sgn(k)k^\prime}{j}\\
&=\jacobi{2^n \sgn(k) k^\prime}{j}\\
&=\jacobi{k}{j}.
\end{align*}
\item $k=0$. This case is trivial, although we do not need it for our applications.
\end{enumerate}

(NOTE: In order to make this more clean, this lemma should be put with proofs of the other Jacobi symbol properties)

Continuing with well-definedness, suppose $[i] = [j]$ in $(\mathbf{Z}/D\mathbf{Z})^*$, $i$ and $j$ odd.
\[\chi([i])=\jacobi{D}{i}\quad\text{and}\quad\chi([j])=\jacobi{D}{j}.\] Then $(D/i)=(D/j)$ is a direct application of the lemma.

Next we show that $\chi$ is total; specifically, for every $[i]\in (\mathbf{Z}/D\mathbf{Z})^*$, there is some positive odd $j$ such that $[j] = [i]$.
  Case: $i < 0$. $i+(-i)|D|$ and $i+(-i+1)|D|$ are both nonnegative and are in the same class as $[i]$. If $D$ is odd, they have opposite parity so let $j$ be the odd one. If $D$ is even, $i$ must be odd since $\gcd(i,D)=1$, so $j$ can be either one.
  Case: $i > 0$. Do the same as the previous case but with candidates $i+|D|$ and $i+2|D|$.

The map $\chi$ is a homomorphism since, assuming $a,b$ odd, $\chi([a][b]) = \chi([ab]) = (D/ab) = (\frac{D}{a})\cdot(\frac{D}{b}) = \chi([a])\cdot\chi([b])$ using a basic property of Jacobi symbols.

Now suppose we have a (possibly different) homomorphism
$\phi:(\mathbf{Z}/D\mathbf{Z})^*\to\{\pm 1\}$, subject to the constraint that for any positive odd prime $p$ not dividing $D$ we have
  $\phi([p]) = (\frac{D}{p})$.

For any $[i]\in (\mathbf{Z}/D\mathbf{Z})^*$, we showed that there is a positive odd integer $j$ such that $[i] = [j]$. This means we can write $j=p_1^{a_1}\cdot\ldots\cdot p_n^{a_n}$ where each $p_i$ is a positive odd prime. Then
\begin{align*}
\phi([j]) &= \phi([p_1^{a_1}\cdots p_n^{a_n}])\\
       &= \phi([p_1])^{a_1}\cdots \phi([p_n])^{a_n}\\
       &= \jacobi{D}{p_1}^{a_1}\cdots\jacobi{D}{p_n}^{a_n}\\
       &= \jacobi{D}{p_1^{a_1}\cdots p_n^{a_n}}\\
       &= \jacobi{D}{j} = \chi([j]).
\end{align*}
Thus $\chi$ is the \emph{unique} homomorphism
$(\mathbf{Z}/D\mathbf{Z})^* \mapsto {\pm 1}$
such that $\chi([p]) = (D/p)$ for positive odd primes $p\nmid D$.

We can calculate some values of $\chi$ explicitly, namely $-1$ and 2.

If $D>0$ and $D\equiv 1\bmod 4$ then $2D-1$ is odd and positive. Then
\[\chi([-1])=\chi([2D-1])=\jacobi{D}{2D-1}=\jacobi{2D-1}{D}=\jacobi{-1}{D}=1.\]

If $D>0$ and $D\equiv 0\bmod 4$ then $D-1$ is odd and positive. Then
\[\chi([-1]) = \chi([D-1]) = \jacobi{D}{D-1} = \jacobi{1}{D-1} = 1.\]

If $D<0$ and $D\equiv 1\bmod 4$ then $-2D-1$ is odd and positive. Then
\begin{align*}
  \chi([-1]) &= \chi([-2D-1])\\
             &= \jacobi{D}{-2D-1}\\
             &= \jacobi{-D}{-2D-1}\jacobi{-1}{-2D-1}\\
             &= \jacobi{-1}{D}\cdot(-1)^\frac{(D+1)(D+1)}{2}\jacobi{-1}{-2D-1}\\
             &= -1\cdot 1\cdot 1 = -1.
\end{align*}

If $D<0$ and $D\equiv 0\bmod 4$ then $-D-1$ is odd and positive. Then
\[\jacobi{D}{-D-1} = \jacobi{-1}{-D-1}\jacobi{-D}{-D-1} = (-1)\jacobi{1}{-D-1} = -1.\]

Summarizing,
\[\chi([-1]) =
  \begin{cases}
    1&\quad D>0\\
    -1&\quad D<0.
  \end{cases}
\]

For $\chi([2])$, since 2 is even we must have $D$ odd (so $D\equiv 1\bmod 4$).

If $D>0$, $2+D$ is odd and positive.
\[\jacobi{D}{2+D} = \jacobi{2+D}{D} = \jacobi{2}{D} = 1, D\equiv 1\bmod 8\text{ or }-1, D\equiv 5\bmod 8.\]

If $D<0$, $2{-}D$ is odd and positive.
\begin{align*}
\jacobi{D}{2{-}D} = \jacobi{2}{2{-}D} &= 1\text{ when }2{-}D=1\bmod 8 (\implies D\equiv 1\bmod 8)\\
&=-1\text{ when }2{-}D\equiv 5\bmod 8 (\implies D\equiv 5\bmod 8).
\end{align*}

Summarizing,
\[\chi([2]) =
  \begin{cases}
    1& D\equiv 1\bmod 8\\
    -1& D\equiv 5\bmod 8.
  \end{cases}
\]

One benefit of this homomorphism $\chi$ is that it gives an equivalent way of thinking about the reciprocity step in representing $x^2 + ny^2$.
Remember $p\mid x^2+ny^2$, $\gcd(x,y)=1 \iff (\frac{-n}{p})=1$.
This is almost the map $\chi$, except there are no restrictions on $n$ whereas $\chi$ needs the modulus to be 0 or $1\bmod 4$.
The trick is to notice that $(\frac{-n}{p})=1 \iff (-4n/p)=1$; now the set of primes $p$ dividing $x^2+ny^2$ are exactly the equivalence classes $[p]$ for which $\chi([p])=1$ (with $D=-4n$).

The book then gives a characterization of $\ker(x)$ which I don't see the significance of (besides having historical context) but maybe this will change as I read more.

\end{document}

\documentclass[b5paper,12pt,oneside,openright]{memoir}
\usepackage{amsmath, amsthm, amssymb}
\usepackage{enumitem}
\usepackage{color}

\nonzeroparskip

\setlength{\uppermargin}{22mm}
\setlength{\spinemargin}{22mm}
\setlength{\textheight}{206mm}
\setlength{\textwidth}{132mm}
\setlength{\oddsidemargin}{0mm}
\fixthelayout

\newtheorem{thm}{Theorem}
\newtheorem*{thm*}{Theorem}

\DeclareMathOperator{\sgn}{sgn}
\newcommand{\jacobi}[2]{\left(\frac{#1}{#2}\right)}
\newcommand{\Zmod}[1]{\mathbf{Z}/#1\mathbf{Z}}

\begin{document}
\section{$2n$ points in general position are the zeroes of quadratic polynomials}

$\Gamma\subseteq\P^n$ is a collection of $2n$ points. Given any decomposition $\Gamma=\Gamma_1\cup\Gamma_2$ with $|\Gamma_1|=|\Gamma_2|=n$, we must have $\Gamma_1$ and $\Gamma_2$ spanning hyperplanes $\Lambda_1$ and $\Lambda_2$ if the points of $\Gamma$ are in general position. Then there are homogeneous linear polynomials vanishing on $\Lambda_1$ and $\Lambda_2$ whose product is a homogeneous quadratic polynomial vanishing on $\Lambda_1\cup\Lambda_2$.

Now let $S$ be the set of all (quadratic) polynomials obtained in this way from some partition of $\Gamma$ into two sets of size $n$ and consider a point $q\in\P^n$ with the property that $F(q)=0$ for all $F\in S$. The goal is to show that $q\in\Gamma$.

Choose a set $M\subseteq\Gamma$ of size $n$ such that $q$ is in the hyperplane spanned by $M$. Fix $m\in M$ and consider the hyperplane spanned by $M\setminus\{m\}$ and any point $p\in\Gamma\setminus M$; call this hyperplane $\Lambda_p$. Then $m$ and the other $n-1$ points of $\Gamma\setminus M$ span a hyperplane $\Lambda_p^\prime$. Note that this gives us a homogeneous quadratic polynomial $F_p\in S$ whose zero locus is exactly $\Lambda_p\cup\Lambda_p^\prime$.

$m$ cannot be in $\Lambda_p$, otherwise we would have $n+1$ points in general position (specifically $M\cup\{p\}$) spanning a hyperplane. Since $q$ is in the span of $M$, it follows that $q$ cannot be in $\Lambda_p$. But $F_p(q)=0$, so $q\in \Lambda_p\cup\Lambda_p^\prime$. Thus we conclude that $q\in\Lambda_p^\prime$.

There were $n$ choices for the point $p\in\Gamma\setminus M$; each choice yields a distinct hyperplane $\Lambda_p^\prime$ and each such hyperplane contains the points $m$ and $q$. Then
\[
\bigcap_{p\in\Gamma\setminus M}\Lambda_p^\prime
\]
is a single point, as it is the intersection of $n$ distinct hyperplanes in $\P^n$. Since $m$ and $q$ are both in this intersection, $m=q$. In particular, $q\in\Gamma$, so the zero locus of the quadratic polynomials in $S$ is exactly $\Gamma$.

\section{$kn$ points in general position are the zeroes of degree $k$ polynomials}

I can't seem to get the previous argument to generalize easily for $k>2$. The case $k=2$ was easy because we constructed a family of $n$ pairs of hyperplanes and were able to show that two points were not in one of the hyperplanes (for every pair), meaning they must both be in the other.

For $k>2$, the analogous construction gives a family of $n$ unions of $k$ hyperplanes. We can still show by generality that the two points are not in one of the hyperplanes, but I can't seem to show that they need to be in the same of the remaining hyperplanes. Maybe a different type of argument is needed.

\end{document}

\documentclass[b5paper,12pt,oneside,openright]{memoir}
\usepackage{amsmath, amsthm, amssymb}
\usepackage{enumitem}
\usepackage{color}

\nonzeroparskip

\setlength{\uppermargin}{22mm}
\setlength{\spinemargin}{22mm}
\setlength{\textheight}{206mm}
\setlength{\textwidth}{132mm}
\setlength{\oddsidemargin}{0mm}
\fixthelayout

\newtheorem{thm}{Theorem}
\newtheorem*{thm*}{Theorem}
\newtheorem{lem}{Lemma}
\newtheorem*{lem*}{Lemma}

\DeclareMathOperator{\sgn}{sgn}
\newcommand{\jacobi}[2]{\left(\frac{#1}{#2}\right)}
\newcommand{\Zmod}[1]{\mathbf{Z}/#1\mathbf{Z}}

\begin{document}
\section{Thoughts}
Let $n$ be a positive integer and $p$ an odd prime with $\gcd(n,p)=1$. Now assume $(-n/p)=1$. By corollary 2.6 we can find $f(x,y)$ with discriminant $-4n$ which represents $p$. But every equivalence class of quadratic forms has a unique reduced representative, so there exists $\hat{f}(x,y)$ which is reduced, properly equivalent to $f$ (so has discriminant $-4n$), and represents $p$ by a result from the previous section [cite].

Conversely, suppose $f(x,y)$ is a reduced form of discriminant $-4n$. It is properly equivalent to a form $px^2 + bxy + cy^2$ for some $b,c\in\mathbf{Z}$, which as discriminant $-4n = b^2-4pc$. This shows $-4n\equiv b^2\bmod p$, i.e. $(-4n/p)=1$ and thus $(-n/p)=1$.

After getting to this point in the book, I finally realize the benefit of the homomorphism $\chi$ from section 1. Better late than never. Here's my current take:

If we're given a positive integer $n$, we want to be able to say \emph{exactly} which primes $p$ are represented by $x^2 + ny^2$. Section 1 showed that this question is related to the quadratic character of $-n\bmod p$. Lemma 1.7 says that $(-n/p)=1$ is necessary for $p$ to be represented by $x^2+ny^2$, and we can prove that it is sufficient using ad hoc methods in some cases ($n=1, 2$ and 3). It was also shown that it's \emph{not} sufficient in general ($n=5$).

Then in section 2A we found that the theory of quadratic forms is a good language to reframe our question. This is because $(-n/p)=1$, our necessary condition, is equivalent to representing $p$ by a reduced form of discriminant $-4n$. This is good for two reasons:

\begin{enumerate}
\item $x^2+ny^2$ is one of those reduced forms. In fact, this lets us tighten our condition \emph{too} much and get a sufficient, but not necessary, condition. If there are no other reduced forms of discriminant $-4n$, we are done. This only happens when $n=1,2,3,4$ or 7.

\item We now have a tangible way to attack the ``not sufficient'' part of our $(-n/p)$ condition. Specifically, if we can develop some machinery which allows us to distinguish primes which are representable by different reduced forms of the same discriminant, we can get closer to an exact characterisation of primes representable by $x^2+ny^2$.
\end{enumerate}

How does this relate to $\chi$? Well, say we develop more theory and you ask, for a certain $n$, exactly which primes are representable by $x^2+ny^2$. Then I say, ``Easy, it's just the primes $p$ for which $(-n/p)=1$ and (a bunch of other stuff about quadratic forms and whatever is in the rest of the book)''.

\emph{A priori}, this might not be super helpful. The things that we showed about $\chi$ allow us to rephrase the reciprocity condition in terms of congruence classes modulo $-4n$, and presumably the conditions we add later will tell us how to partition those congruence classes further. I think this is the point of corollary 1.19 and the subsequent discussion of the characterisation of $\ker(\chi)$.

\section{When is $h(-4n)=1$ for $n>0$?}

$x^2+ny^2$ is reduced for all positive $n$ since $|b|=0< 1\leq n$ and $b\geq 0$ so we don't care if $a=c$.

\subsection{Case: $n$ is not a prime power}

Write $n=a\cdot c$ with $\gcd(a,c)=1$ (neither $a$ nor $c$ is 1, or else we're recounting the forms above). Then one of $ax^2+cy^2$ or $cx^2+ay^2$ is reduced, and both have discriminant $-4ac=-4n$. So in this case $h(-4n)>1$.

\subsection{Case: $n$ is a power of 2 and $n\geq 16$}

For $r\geq 2$, the quadratic form
\[4x^2+4xy+(2^r+1)y^2\]
is reduced since $4\leq 4\leq 2^r+1$ and $b\geq 0$. The discriminant is
\[16-4\cdot 4\cdot (2^r+1)=16(1-(2^r+1))=-16\cdot 2^r=-4\cdot 2^{r+2}.\]
So $h(-4n)>1$ when $n\geq 2^{2+2}=16$ is a power of 2.

\subsection{Case: $n+1$ is a power of 2 and $n\geq 63$}

For $s\geq 3$, the quadratic form
\[8x^2+6xy+(2^s+1)y^2\]
is reduced since $6\leq 8\leq 2^3+1$ and $b\geq 0$. The discriminant is
\[36-32(2^s+1)=4-2^{s+5}=-4(2^{s+3}-1).\]
So $h(-4n)>1$ when $n\geq 2^{3+3}-1=63$ and $n+1$ is a power of 2. There is some overlap here with the first case but it doesn't affect the result. If we wanted to make \emph{disjoint} cases which cover all positive integers we could restrict $n$ to be an odd prime power. This is what the book does.

\subsection{Case: $n+1$ is even but not a power of 2}

Write $n+1 = ac$ where $a = 2^m$ and $c$ is odd. If $a<c$, $ax^2+2xy+cy^2$ is reduced and $cx^2+2xy+ay^2$ is reduced if $c<a$. In either case the discriminant is
\[4-4ac=-4(ac-1)=-4n,\]
so $h(-4n)>1$ in this case.

\subsection{Case: $n$ is not in one of the above cases}

The only cases not covered above are powers of 2 less than 16 and odd prime powers less than 63 whose successor is a power of two. Specifically, we're left with
\[n=1, 2, 4, 8\quad\textrm{and}\quad n=3, 7, 31.\]

For $n=8$, $3x^2+2xy+3y^2$ is reduced, so $h(-4\cdot 8)>1$.

For $n=31$, $5x^2\pm 4xy+7y^2$ are reduced, so $h(-4\cdot 31)>2$.

For a reduced form, $a\leq\sqrt{4n/3}$.

For $n=1$, $a\leq 1\implies a=1$ and $4=4c-b^2\implies b=0, c=1$.

For $n=2$, $a\leq 1\implies a=1$ and $8=4c-b^2\implies b=0, c=2$.

For $n=3$, $a\leq 2$, and
\begin{align*}
12=4c-b^2&\implies b\neq 1,2,\textrm{ so }b=0,c=3.\\
12=8c-b^2&\implies b\neq 0,1,\textrm{ but }a=2, b=2, c=2\textrm{ is not primitive.}
\end{align*}

For $n=4$, $a\leq 2$, and
\begin{align*}
16=4c-b^2&\implies b\neq 1,2,\textrm{ so }b=0,c=4.\\
16=8c-b^2&\implies b\neq 1,2\textrm{ but }a=2, b=0, c=2\textrm{ is not primitive.}\\
\end{align*}

For $n=7$, $a\leq 3$, and
\begin{align*}
28=4c-b^2&\implies b\neq 1,2,3,\textrm{ so }&b=0,c=7.\\
28=8c-b^2&\implies b\neq 0,1,3\textrm{ but }&a=2, b=2, c=4\textrm{ is not primitive.}\\
28=12c-b^2&\implies b\neq 0,1,2,3.\\
\end{align*}

This proves $h(-4n)=1$, $n>0\iff n=1, 2, 3, 4, 7$.

\end{document}
